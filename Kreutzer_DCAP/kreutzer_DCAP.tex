%% Author_tex.tex
%% V1.0
%% 2012/13/12
%% developed by Techset
%%
%% This file describes the coding for rsproca.cls

\documentclass[]{cik}%%%%where rsos is the template name


\usepackage{lineno}
\linenumbers

\usepackage[T1]{fontenc}
\usepackage[utf8]{inputenc}

% Pandoc citation processing
% Pandoc citation processing
\newlength{\cslhangindent}
\setlength{\cslhangindent}{1.5em}
\newlength{\csllabelwidth}
\setlength{\csllabelwidth}{3em}
\newlength{\cslentryspacingunit} % times entry-spacing
\setlength{\cslentryspacingunit}{\parskip}
% for Pandoc 2.8 to 2.10.1
\newenvironment{cslreferences}%
  {}%
  {\par}
% For Pandoc 2.11+
\newenvironment{CSLReferences}[2] % #1 hanging-ident, #2 entry spacing
 {% don't indent paragraphs
  \setlength{\parindent}{0pt}
  % turn on hanging indent if param 1 is 1
  \ifodd #1
  \let\oldpar\par
  \def\par{\hangindent=\cslhangindent\oldpar}
  \fi
  % set entry spacing
  \setlength{\parskip}{#2\cslentryspacingunit}
 }%
 {}
\usepackage{calc} % for calculating minipage widths
\newcommand{\CSLBlock}[1]{#1\hfill\break}
\newcommand{\CSLLeftMargin}[1]{\parbox[t]{\csllabelwidth}{#1}}
\newcommand{\CSLRightInline}[1]{\parbox[t]{\linewidth - \csllabelwidth}{#1}}
\newcommand{\CSLIndent}[1]{\hspace{\cslhangindent}#1}

\usepackage{caption}
\usepackage{booktabs}
\usepackage{longtable}
\usepackage{array}
\usepackage{multirow}
\usepackage{wrapfig}
\usepackage{float}
\usepackage{colortbl}
\usepackage{pdflscape}
\usepackage{tabu}
\usepackage{threeparttable}
\usepackage{threeparttablex}
\usepackage[normalem]{ulem}
\usepackage{makecell}
\usepackage{xcolor}

%%%% *** Do not adjust lengths that control margins, column widths, etc. ***

%%%%%%%%%%% Defining Enunciations  %%%%%%%%%%%
\newtheorem{theorem}{\bf Theorem}[section]
\newtheorem{condition}{\bf Condition}[section]
\newtheorem{corollary}{\bf Corollary}[section]
%%%%%%%%%%%%%%%%%%%%%%%%%%%%%%%%%%%%%%%%%%%%%%%

\begin{document}

%%%% Article title to be placed here
\title{Ketogenic and High-Carbohydrate Diets in Cyclists and
Triathletes: Performance Indicators and Methodological Considerations
From a Pilot Study}

\author{
Andreas Kreutzer$^{1}$,
Austin J. Graybeal$^{1}$,
Petra P. Rack$^{1}$,
Kamiah Moss$^{1}$,
Garrett R. Augsburger$^{1}$,
Jada L. Willis$^{2}$,
Robyn Braun-Trocchio$^{1}$,
Meena Shah$^{1}$}

\address{
  $^{1}$Department of Kinesiology, Harris College of Nursing \& Health
Sciences, Texas Christian University, Fort Worth, TX\\
  $^{2}$Department of Nutritional Sciences, College of Science \&
Engineering, Texas Christian University, Fort Worth, TX}
%%%% Subject entries to be placed here %%%%
\subject{
Physiology \& Nutrition}

%%%% Keyword entries to be placed here %%%%
\keywords{
Endurance Performance,
Fat Adaptation,
Muscle Glycogen}

%%%% Insert corresponding author and its email address}
\corres{
  A. Kruetzer \& M. Shah\\
{\footnotesize \href{mailto:m.shah@tcu.edu}{\nolinkurl{m.shah@tcu.edu}};
\href{mailto:a.kreutzer@tcu.edu}{\nolinkurl{a.kreutzer@tcu.edu}}}
}
\inserttype{
{\Large Research}
}

\insertcite{
  {10.51224/XXXXXXXXX}
}


\editor{
  John P. Mills
}

%%%% Abstract text to be placed here %%%%%%%%%%%%
\begin{abstract}
Endurance athletes frequently employ nutritional strategies to enhance
performance. While professional organizations recommend high
carbohydrate diets to maximize performance, many athletes, and
researchers have recently shown renewed interest in the ketogenic diet
in hopes to promote ``fat adaptation'', which would allow athletes to
make use of the essentially unlimited energy resources from stored body
fat. This would circumvent one fatigue mechanism, the depletion of
muscle glycogen stores, that has been considered central to performance
outcomes in endurance events. The present study investigated the effects
of participants' habitual diet, high carbohydrate diet, and ketogenic
diet on endurance performance in a 30-km simulated cycling time trial,
physiological responses during the time trial, and muscle session fuel
percentile before and after the time trial using ultrasonic imaging. Due
to the COVID-19 pandemic, data collection ceased after only six
recreational cyclists and triathletes (f = 4, m = 2; age: 37.2
{[}12.2{]}; VO2max: 46.8 {[}6.8{]} ml/kg/min; weekly cycling distance:
225.3 {[}64.2{]} km). Due to the small sample size, we do not report
inferential statistics for our primary outcome measure, cycling
performance. Participants produced the lowest mean power output during
the time trial following the ketogenic diet (172 +/- 93 W) and the
highest mean power output following the high carbohydrate diet (200
{[}92{]} W). Oxygen consumption, heart rate, and perceived exertion
during the time trial were similar in all conditions. Fat oxidation
rates were highest in the ketogenic diet condition (0.62 {[}0.11{]}
g/min) and lowest in the high carbohydrate condition (0.14 {[}0.11{]}
g/min). Session fuel percentile was lower following the ketogenic diet
compared with the habitual diet (Mean Difference = -10.0 {[}12.7{]} \%)
and lower following the time trial compared with fasted resting values
across all conditions. We discuss methodological considerations into the
use of exercise equipment, nutritional interventions, and statistical
analysis strategies for study designs like the present. Further research
is needed to assess the impact of high carbohydrate and ketogenic diets
on time trial performance in this population. ClinicalTrials.gov
Identifier: NCT04097171; OSF preregistration:
\url{https://osf.io/ujx6e/}
\end{abstract}
%%%%%%%%%%%%%%%%%%%%%%%%%%%

%% Some pieces required from the pandoc template
\providecommand{\tightlist}{%
  \setlength{\itemsep}{0pt}\setlength{\parskip}{0pt}}
\providecommand{\EndFirstPage}{%
}

\definecolor{jobcolor}{cmyk}{0,0,0,.95}
\definecolor{joblightcolor}{cmyk}{0,0,0,.95}
\definecolor{abstractcolor}{RGB}{207, 250, 209}
\definecolor{copyrightcolor}{RGB}{207, 250, 209}

\maketitle

\newpage

\captionsetup[table]{labelformat=empty}
\captionsetup[figure]{labelformat=empty}
\raggedbottom

\hypertarget{introduction}{%
\section{Introduction}\label{introduction}}

Nutritional interventions remain at the forefront of strategies employed
by athletes to enhance their performance
(\protect\hyperlink{ref-1}{Thomas et al., 2016}). Commonly employed
approaches among endurance athletes include a high daily intake of
dietary carbohydrate (6-10 g/kg/day) and carbohydrate loading (10-12
g/kg/day) before an event, since low muscle glycogen is a
well-established cause of fatigue (\protect\hyperlink{ref-2}{Jeukendrup,
2004}, \protect\hyperlink{ref-3}{2011}). Contrary to this traditionally
favored strategy, endurance athletes and researchers have recently began
expressing increased interest in a low carbohydrate, high fat ketogenic
diet again (\protect\hyperlink{ref-4}{L. Burke, 2017}). When following a
ketogenic diet, athletes typically limit their carbohydrate intake to
\textless50 g or 5-10\% of their total daily energy intake
(\protect\hyperlink{ref-5}{Feinman et al., 2015}). The proposed benefit
of this diet approach is ``fat adaptation'', enabling the oxidation of
fat as the main energy substrate at exercise intensities
(e.g.~\textgreater70\% of maximal oxygen consumption
\(\dot{V}O_{2}max\)) where the oxidation of carbohydrate would typically
predominate (\protect\hyperlink{ref-6}{Carey et al., 1985};
\protect\hyperlink{ref-7}{Lambert et al., 1994},
\protect\hyperlink{ref-8}{1997}). This would essentially create
unlimited energy resources, as the body can store more than 74,000 kcal
in subcutaneous, visceral, and intramuscular fat
(\protect\hyperlink{ref-9}{Kenney et al., 2020}). Despite its recent
resurgence in popularity, the ketogenic diet's restrictive nature
counters the current dietary recommendations of several professional
organizations, which state that low carbohydrate availability before
exercise is a significant component of diminished exercise capacity and
performance (\protect\hyperlink{ref-10}{L. M. Burke et al., 2019};
\protect\hyperlink{ref-11}{Kerksick et al., 2017};
\protect\hyperlink{ref-1}{Thomas et al., 2016}). ̇VO\textsubscript{2}max

Phasellus viverra nulla ut metus varius laoreet. Quisque rutrum. Aenean
imperdiet. Etiam ultricies nisi vel augue. Curabitur ullamcorper
ultricies nisi. Nam eget dui. Etiam rhoncus. Maecenas tempus, tellus
eget condimentum rhoncus, sem quam semper libero, sit amet adipiscing
sem neque sed ipsum. Nam quam nunc, blandit vel, luctus pulvinar,
hendrerit id, lorem. Maecenas nec odio et ante tincidunt tempus. Donec
vitae sapien ut libero venenatis faucibus. Nullam quis ante. Etiam sit
amet orci eget eros faucibus tincidunt. Duis leo. Sed fringilla mauris
sit amet nibh. Donec sodales sagittis magna. Sed consequat, leo eget
bibendum sodales, augue velit cursus nunc, quis gravida magna mi a
libero. Fusce vulputate eleifend sapien.

Vestibulum purus quam, scelerisque ut, mollis sed, nonummy id, metus.
Nullam accumsan lorem in dui. Cras ultricies mi eu turpis hendrerit
fringilla. Vestibulum ante ipsum primis in faucibus orci luctus et
ultrices posuere cubilia Curae; In ac dui quis mi consectetuer lacinia.
Nam pretium turpis et arcu. Duis arcu tortor, suscipit eget, imperdiet
nec, imperdiet iaculis, ipsum. Sed aliquam ultrices mauris. Integer ante
arcu, accumsan a, consectetuer eget, posuere ut, mauris. Praesent
adipiscing. Phasellus ullamcorper ipsum rutrum nunc. Nunc nonummy metus.
Vestibulum volutpat pretium libero. Cras id dui. Aenean ut eros et nisl
sagittis vestibulum. Nullam nulla eros, ultricies sit amet, nonummy id,
imperdiet feugiat, pede. Sed lectus. Donec mollis hendrerit risus.
Phasellus nec sem in justo pellentesque facilisis. Etiam imperdiet
imperdiet orci. Nunc nec neque. Phasellus leo dolor, tempus non, auctor
et, hendrerit quis, nisi. Curabitur ligula sapien, tincidunt non,
euismod vitae, posuere imperdiet, leo. Maecenas malesuada. Praesent
congue erat at massa. Sed cursus turpis vitae tortor. Donec posuere
vulputate arcu. Phasellus accumsan cursus velit. Vestibulum ante ipsum
primis in faucibus orci luctus et ultrices posuere cubilia Curae; Sed
aliquam, nisi quis porttitor congue, elit erat euismod orci, ac

\newpage

\hypertarget{methods}{%
\section{Methods}\label{methods}}

\hypertarget{greek}{%
\subsection{Greek}\label{greek}}

Many times greek letters/symbols need to be provided outside of math
mode. So you may say \(\beta\)

\hypertarget{superscript}{%
\subsection{Superscript}\label{superscript}}

You can give Superscript\textsuperscript{1} or
Subscript\textsubscript{2}

\hypertarget{quotes-and-block-quotes}{%
\subsection{Quotes and Block Quotes}\label{quotes-and-block-quotes}}

\begin{quote}
This can easily be done

\begin{itemize}
\tightlist
\item
  ME
\end{itemize}
\end{quote}

\hypertarget{links}{%
\subsection{Links}\label{links}}

\begin{verbatim}
A [linked phrase][id].
\end{verbatim}

At the bottom of the document:

\begin{verbatim}
[id]: http://example.com/ "Title"
\end{verbatim}

\hypertarget{images}{%
\subsection{Images}\label{images}}

\begin{verbatim}
![alt text][id]
\end{verbatim}

\hypertarget{math}{%
\subsection{Math}\label{math}}

Fortunately the math formulas do not differ too much for HTML and PDF
documents. For inline math a single \texttt{\$} is necessary while
\texttt{\$\$} creates formula on its own line.

\[
BSIc \space (mg^2/mm^4) = ToD^2 \space (mg/cm^3/1000) \cdot ToA.tb^2 \space (mm^2) \space 
\]

\newpage

\hypertarget{results}{%
\section{Results}\label{results}}

Lorem ipsum dolor sit amet, consectetuer adipiscing elit. Aenean commodo
ligula eget dolor. Aenean massa. Cum sociis natoque penatibus et magnis
dis parturient montes, nascetur ridiculus mus. Donec quam felis,
ultricies nec, pellentesque eu, pretium quis, sem. Nulla consequat massa
quis enim. Donec pede justo, fringilla vel, aliquet nec, vulputate eget,
arcu.

\begin{figure}[H]
\includegraphics[width=1\linewidth]{STORK_overlay_blk} \caption{Figure 1: Somtimes greek in captions as well $\beta$ but make sure to use double backslash}\label{fig:fig1pdf}
\end{figure}

In enim justo, rhoncus ut, imperdiet a, venenatis vitae, justo. Nullam
dictum felis eu pede mollis pretium. Integer tincidunt. Cras dapibus.
Vivamus elementum semper nisi. Aenean vulputate eleifend tellus. Aenean
leo ligula, porttitor eu, consequat vitae, eleifend ac, enim. Aliquam
lorem ante, dapibus in, viverra quis, feugiat a, tellus. Phasellus
viverra nulla ut metus varius laoreet. Quisque rutrum. Aenean imperdiet.
Etiam ultricies nisi vel augue. Curabitur ullamcorper ultricies nisi.
Nam eget dui. Etiam rhoncus. Maecenas tempus, tellus eget condimentum
rhoncus, sem quam semper libero, sit amet adipiscing sem neque sed
ipsum. Nam quam nunc, blandit vel, luctus pulvinar,

\newpage

\begin{ThreePartTable}
\begin{TableNotes}
\item \textit{Note.} 
\item x = note 1; y = note 2.
\end{TableNotes}
\begin{longtable}[t]{lrr}
\caption{\label{tab:unnamed-chunk-2}\textbf{Table 1}:Example.}\\
\toprule
  & MPG & Cylinder\\
\midrule
Mazda RX4 & 21 & 6\\
\addlinespace
Mazda RX4 Wag & 21 & 6\\
\bottomrule
\insertTableNotes
\end{longtable}
\end{ThreePartTable}

\hypertarget{discussion}{%
\section{Discussion}\label{discussion}}

Lorem ipsum dolor sit amet, consectetuer adipiscing elit. Aenean commodo
ligula eget dolor. Aenean massa. Cum sociis natoque penatibus et magnis
dis parturient montes, nascetur ridiculus mus. Donec quam felis,
ultricies nec, pellentesque eu, pretium quis, sem. Nulla consequat massa
quis enim. Donec pede justo, fringilla vel, aliquet nec, vulputate eget,
arcu. In enim justo, rhoncus ut, imperdiet a, venenatis vitae, justo.
Nullam dictum felis eu pede mollis pretium. Integer tincidunt. Cras
dapibus. Vivamus elementum semper nisi. Aenean vulputate eleifend
tellus.

Aenean leo ligula, porttitor eu, consequat vitae, eleifend ac, enim.
Aliquam lorem ante, dapibus in, viverra quis, feugiat a, tellus.
Phasellus viverra nulla ut metus varius laoreet. Quisque rutrum. Aenean
imperdiet. Etiam ultricies nisi vel augue. Curabitur ullamcorper
ultricies nisi. Nam eget dui. Etiam rhoncus. Maecenas tempus, tellus
eget condimentum rhoncus, sem quam semper libero, sit amet adipiscing
sem neque sed ipsum. Nam quam nunc, blandit vel, luctus pulvinar,
hendrerit id, lorem. Maecenas nec odio et ante tincidunt tempus.

Donec vitae sapien ut libero venenatis faucibus. Nullam quis ante. Etiam
sit amet orci eget eros faucibus tincidunt. Duis leo. Sed fringilla
mauris sit amet nibh. Donec sodales sagittis magna. Sed consequat, leo
eget bibendum sodales, augue velit cursus nunc, quis gravida magna mi a
libero. Fusce vulputate eleifend sapien. Vestibulum purus quam,
scelerisque ut, mollis sed, nonummy id, metus. Nullam accumsan lorem in
dui. Cras ultricies mi eu turpis hendrerit fringilla. Vestibulum ante
ipsum primis in faucibus orci luctus et ultrices posuere cubilia Curae;
In ac dui quis mi consectetuer lacinia.

Nam pretium turpis et arcu. Duis arcu tortor, suscipit eget, imperdiet
nec, imperdiet iaculis, ipsum. Sed aliquam ultrices mauris. Integer ante
arcu, accumsan a, consectetuer eget, posuere ut, mauris. Praesent
adipiscing. Phasellus ullamcorper ipsum rutrum nunc. Nunc nonummy metus.
Vestibulum volutpat pretium libero. Cras id dui. Aenean ut eros et nisl
sagittis vestibulum. Nullam nulla eros, ultricies sit amet, nonummy id,
imperdiet feugiat, pede. Sed lectus. Donec mollis hendrerit risus.
Phasellus nec sem in justo pellentesque facilisis. Etiam imperdiet
imperdiet orci. Nunc nec neque. Phasellus leo dolor, tempus non, auctor
et, hendrerit quis, nisi. Curabitur ligula sapien, tincidunt non,
euismod vitae, posuere imperdiet, leo. Maecenas malesuada. Praesent
congue erat at massa. Sed cursus turpis vitae tortor. Donec posuere
vulputate arcu. Phasellus accumsan cursus velit. Vestibulum ante ipsum
primis in faucibus orci luctus et ultrices posuere cubilia Curae; Sed
aliquam, nisi quis porttitor congue, elit erat euismod orci, ac

\hypertarget{conclusion}{%
\subsection{Conclusion}\label{conclusion}}

orem ipsum dolor sit amet, consectetuer adipiscing elit. Aenean commodo
ligula eget dolor. Aenean massa. Cum sociis natoque penatibus et magnis
dis parturient montes, nascetur ridiculus mus. Donec quam felis,
ultricies nec, pellentesque eu, pretium quis, sem. Nulla consequat massa
quis enim. Donec pede justo, fringilla vel, aliquet nec, vulputate eget,
arcu. In enim justo, rhoncus ut, imperdiet a, venenatis vitae, justo.
Nullam dictum felis eu pede mollis pretium. Integer tincidunt. Cras
dapibus. Vivamus elementum semper nisi. Aenean vulputate eleifend
tellus. Aenean leo ligula, porttitor eu, consequat vitae, eleifend ac,
enim. Aliquam lorem ante, dapibus in, viverra quis, feugiat a.

\newpage

\hypertarget{additional-information}{%
\section{Additional Information}\label{additional-information}}

\hypertarget{data-accessibility}{%
\subsection{Data Accessibility}\label{data-accessibility}}

Data is available via a \ldots.

\hypertarget{author-contributions}{%
\subsection{Author Contributions}\label{author-contributions}}

\begin{itemize}
\tightlist
\item
  Contributed to conception and design:
\item
  Contributed to acquisition of data:
\item
  Contributed to analysis and interpretation of data:
\item
  Drafted and/or revised the article:
\item
  Approved the submitted version for publication:
\end{itemize}

\hypertarget{conflict-of-interest}{%
\subsection{Conflict of Interest}\label{conflict-of-interest}}

Authors have no conflicts of interest to declare.

\hypertarget{funding}{%
\subsection{Funding}\label{funding}}

Support provided by \ldots.

\hypertarget{acknowledgments}{%
\subsection{Acknowledgments}\label{acknowledgments}}

We thank everybody.

\hypertarget{preprint}{%
\subsection{Preprint}\label{preprint}}

The pre-publication version of this manuscript can be found on SportRxiv
(DOI: XXXXXXXXXX).

\newpage

\hypertarget{references}{%
\section{References}\label{references}}

\parindent0pt 
\setlength{\parskip}{1em}

\hypertarget{refs}{}
\begin{CSLReferences}{1}{0}
\leavevmode\vadjust pre{\hypertarget{ref-4}{}}%
Burke, L. (2017). Low Carb High Fat (LCHF) Diets for Athletes -- Third
Time Lucky? \emph{J Sci Med Sport}, \emph{20 Suppl 1: S1}.
\url{https://doi.org/ghp7b8}

\leavevmode\vadjust pre{\hypertarget{ref-10}{}}%
Burke, L. M., Castell, L. M., Casa, D. J., Close, G. L., Costa, R. J.
S., Desbrow, B., Halson, S. L., Lis, D. M., Melin, A. K., Peeling, P.,
Saunders, P. U., Slater, G. J., Sygo, J., Witard, O. C., Bermon, S., \&
Stellingwerff, T. (2019). International Association of Athletics
Federations Consensus Statement 2019: Nutrition for Athletics. \emph{Int
J Sport Nutr Exerc Metab}, \emph{29}, 73--84,.
\url{https://doi.org/gjbrp2.}

\leavevmode\vadjust pre{\hypertarget{ref-6}{}}%
Carey, A. L., Staudacher, H. M., Cummings, N. K., Stepto, N. K.,
Nikolopoulos, V., Burke, L. M., \& Hawley, J. A. (1985). Effects of Fat
Adaptation and Carbohydrate Restoration on Prolonged Endurance Exercise.
\emph{J Appl Physiol}, \emph{91}, 115--122,.
\url{https://doi.org/10.1152/jappl.2001.91.1.115.}

\leavevmode\vadjust pre{\hypertarget{ref-5}{}}%
Feinman, R. D., Pogozelski, W. K., Astrup, A., Bernstein, R. K., Fine,
E. J., Westman, E. C., Accurso, A., Frassetto, L., Gower, B. A.,
McFarlane, S. I., Nielsen, J. V., Krarup, T., Saslow, L., Roth, K. S.,
Vernon, M. C., Volek, J. S., Wilshire, G. B., Dahlqvist, A., Sundberg,
R., \ldots{} Worm, N. (2015). Dietary Carbohydrate Restriction As the
First Approach in Diabetes Management. \emph{Critical Review and
Evidence Base. Nutrition}, \emph{31}, 1--13,.
\url{https://doi.org/10.1016/j.nut.2014.06.011.}

\leavevmode\vadjust pre{\hypertarget{ref-2}{}}%
Jeukendrup, A. E. (2004). Carbohydrate Intake During Exercise and
Performance. \emph{Nutrition}, \emph{20}, 669--677,.
\url{https://doi.org/bnpn7g}

\leavevmode\vadjust pre{\hypertarget{ref-3}{}}%
Jeukendrup, A. E. (2011). Nutrition for Endurance Sports: Marathon,
Triathlon, and Road Cycling. \emph{J Sports Sci}, \emph{29 Suppl 1},
91--99,. \url{https://doi.org/b52hvj}

\leavevmode\vadjust pre{\hypertarget{ref-9}{}}%
Kenney, W. L., Wilmore, J. H., \& Costill, D. L. (2020).
\emph{Physiology of Sport and Exercise} (7th ed.). Human Kinetics.

\leavevmode\vadjust pre{\hypertarget{ref-11}{}}%
Kerksick, C. M., Arent, S., Schoenfeld, B. J., Stout, C., JR, B, W., CD,
T., L, K., D, S.-R., AE, K., RB, W., D, A., PJ, V., TA, O., MJ, W., R,
G., M, Z., TN, A., \& A.A. (2017). \emph{Antonio J. International
Society of Sports Nutrition Position Stand: Nutrient Timing. J Int Soc
Sports Nutr}, \emph{14}, 33,. \url{https://doi.org/gg2nh6}

\leavevmode\vadjust pre{\hypertarget{ref-8}{}}%
Lambert, E. V., Hawley, J. A., Goedecke, J., Noakes, T. D., \& Dennis,
S. C. (1997). Nutritional Strategies for Promoting Fat Utilization and
Delaying the Onset of Fatigue During Prolonged Exercise. \emph{J Sports
Sci}, \emph{15}, 315--324,.
\url{https://doi.org/10.1080/026404197367326.}

\leavevmode\vadjust pre{\hypertarget{ref-7}{}}%
Lambert, E. V., Speechly, D. P., Dennis, S. C., \& Noakes, T. D. (1994).
Enhanced Endurance in Trained Cyclists During Moderate Intensity
Exercise Following 2 Weeks Adaptation to a High Fat Diet. \emph{Eur J
Appl Physiol Occup Physiol}, \emph{69}, 287--293,.
\url{https://doi.org/10.1007/bf00392032.}

\leavevmode\vadjust pre{\hypertarget{ref-1}{}}%
Thomas, D. T., Erdman, K. A., \& Burke, L. M. (2016). American College
of Sports Medicine Joint Position Statement. \emph{Nutrition and
Athletic Performance. Med Sci Sports Exerc}, \emph{48}, 543--568,.
\url{https://doi.org/ggnn5v}

\end{CSLReferences}

%
%
%\aucontribute{}

%
%
%
%


\end{document}
